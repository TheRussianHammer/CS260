%
% compile with pdflatex:
%   $ pdflatex homework-hw
% , yields homework-example.pdf
%    You might need to compile twice, if, e.g., you start using references
%

\documentclass[10pt,letterpaper,oneside]{article}
\usepackage[ascii]{inputenc}
\usepackage{amsmath,amsfonts,amssymb}
\usepackage[margin=1in]{geometry}
	\setlength{\parindent}{0em}
	\setlength{\parskip}{1em}

\newtheorem{theorem}{Theorem}

%%%% user definitions %%%%%%%%%%%%%%%%%%%%%%%

\newcommand{\Problem}[1]{\subsection*{Problem #1}}
\newcommand{\Part}[1]{\subsubsection*{Part #1}}
\newcommand{\Solution}{\subsubsection*{Solution}}

	% Forms for Big-Oh notation
\DeclareMathOperator{\Omicron}{O}
\DeclareMathOperator{\omicron}{o}

\newcommand{\BigOh}[1]{\Omicron(#1)}
\newcommand{\LittleOh}[1]{\omicron(#1)}
\newcommand{\BigOmega}[1]{\Omega(#1)}
\newcommand{\LittleOmega}[1]{\omega(#1)}
\newcommand{\BigTheta}[1]{\Theta(#1)}

	% Operators for dominance notation
\newcommand{\domeq}{\sim}
\newcommand{\domle}{\preceq}
\newcommand{\domlt}{\prec}
\newcommand{\domge}{\succeq}
\newcommand{\domgt}{\succ}


%%%%%%%%  You edit stuff below this line  %%%%%%%%%%%%%%%%%%%%%%%%%%

\title{HW1 Theory}
\author{Alexander Kazantsev}
%\date{}  % uses today, by default

\begin{document}

\maketitle

\Problem{1}
	
	Let:
	\begin{eqnarray*}
		\text{$f_1 = n^2$}\\
		\text{$f_2 = n^2 + 1000n$}\\
	\end{eqnarray*}
		\[
			f_3= \left\lbrace%
			\begin{array}{cc}
				n &, \text{$n$ is odd}\\
				n^3 &, \text{$n$ is even}\\
			\end{array} \right.
		\]
		\[
			f_4= \left\lbrace%
			\begin{array}{cc}
				n &, n<=100\\
				n^3 &, n>100\\
			\end{array} \right.
		\]

	\Part{a}
	Statement: $f_1$ and $f_2$ are codominant 
	\Solution
		$f_1 \in \BigOh{f_2}$ = $n^2 \in \BigOh{n^2}$,	
		$c >= n_0$,
		$ n_0 > 0$
		\begin{eqnarray*}
			\text{$n^2 <= cn^2$}\\
			\text{$c=2$}\\
			\text{$n^2 <= 2n^2$}, \forall n>0\\
			\text{$f_2$ does dominate $f_1$}
		\end{eqnarray*}
		$f_2 \in \BigOh{f_1}$ = $n^2 + 1000n \in \BigOh{n^2}$,	
		$c >= n_0$,
		$ n_0 > 0$
		\begin{eqnarray*}
			\text{$n^2 + 1000n <= cn^2$}\\
			\text{$c=10$}\\
			\text{$n^2 + 1000n <= 10n^2$}\\
			\frac{n}{10}\text{$ + 100 <= n$}, \forall n >= 110\\
			\text{$f_1$ does dominate $f_2$ }
		\end{eqnarray*}
	\Part{b}
	Statement: $f_3$ dominates $f_1$, as the upper bound of $f_3$ is $n^3$
	\Solution
		$f_1 \in \BigOh{f_3}$ = $n^2 \in \BigOh{n^3}$,	
		$c >= n_0$,
		$ n_0 > 0$
		\begin{eqnarray*}
			\text{$n^2 <= cn^3$}\\
			\text{$c=1$}\\
			\text{$n^2 <= n^3$}, \forall n>0\\
			\text{$f_3$ does dominate $f_1$ }
		\end{eqnarray*}
		$f_3 \in \BigOh{f_1}$ = $n^3 \in \BigOh{n^2}$,
		$c >= n_0$,
		$ n_0 > 0$
		\begin{eqnarray*}
			\text{$n^3 <= cn^2$}\\
			\text{$c=5$}\\
			\text{$n^3 <= 5n^2$}\\
			\text{$n <= 5$}, \forall n<=5\\
			\text{$f_3$ does not dominate $f_1$ }
		\end{eqnarray*}

	\Part{c}
	Statement: $f_4$ dominates $f_1$, as the upper bound of $f_4$ is $n^3$ 
	\Solution
		$f_1 \in \BigOh{f_4}$ = $n^2 \in \BigOh{n^3}$,	
		$c >= n_0$,
		$ n_0 > 0$
		\begin{eqnarray*}
			\text{$n^2 <= cn^3$}\\
			\text{$c=1$}\\
			\text{$n^2 <= n^3$}\\
			\text{$1 <= n$}, \forall n >= 1\\
			\text{$f_4$ does dominate $f_1$}
		\end{eqnarray*}
		$f_4 \in \BigOh{f_1}$ = $n^3 \in \BigOh{n^2}$,	
		$c >= n_0$,
		$ n_0 > 0$
		\begin{eqnarray*}
			\text{$n^3 <= cn^2$}\\
			\text{$c=10$}\\
			\text{$n^3 <= 10n^2$}\\
			\text{$n <= 10$}, \forall n <= 10\\
			\text{$f_1$ does not dominate $f_3$ }
		\end{eqnarray*}
	\Part{d}
	Statement: $f_3$ dominates $f_2$, as the upper bound of $f_3$ is $n^3$
	\Solution
		$f_2 \in \BigOh{f_3}$ = $n^2 + 1000n \in \BigOh{n^3}$,	
		$c >= n_0$,
		$ n_0 > 0$
		\begin{eqnarray*}
			\text{$n^2 + 1000n<= cn^3$}\\
			\text{$c=10$}\\
			\text{$n^2 + 1000n <= 10n^3$}\\
			\frac{1}{10}\text{$ + \frac{100}{n} <= n$}, \forall n >= 11\\
			\text{$f_3$ does dominate $f_2$}
		\end{eqnarray*}
		$f_3 \in \BigOh{f_2}$ = $n^3\in \BigOh{n^2}$,	
		$c >= n_0$,
		$ n_0 > 0$
		\begin{eqnarray*}
			\text{$n^3<= cn^2$}\\
			\text{$c=10$}\\
			\text{$n^3 <= 10n^2$}\\
			\frac{n}{10}\text{$ <= 1$}, \forall n <= 10\\
			\text{$f_2$ does not dominate $f_3$}
		\end{eqnarray*}
	\Part{e}
	Statement: $f_4$ dominates $f_2$, as the upper bound of $f_4$ is $n^3$
	\Solution
		$f_2 \in \BigOh{f_4}$ = $n^2 + 1000n \in \BigOh{n^3}$,	
		$c >= n_0$,
		$ n_0 > 0$
		\begin{eqnarray*}
			\text{$n^2 + 1000n<= cn^3$}\\
			\text{$c=10$}\\
			\text{$n^2 + 1000n <= 10n^3$}\\
			\frac{1}{10}\text{$ + \frac{100}{n} <= n$}, \forall n >= 11\\
			\text{$f_4$ does dominate $f_2$}
		\end{eqnarray*}
		$f_4 \in \BigOh{f_2}$ = $n^3\in \BigOh{n^2}$,	
		$c >= n_0$,
		$ n_0 > 0$
		\begin{eqnarray*}
			\text{$n^3<= cn^2$}\\
			\text{$c=10$}\\
			\text{$n^3 <= 10n^2$}\\
			\frac{n}{10}\text{$ <= 1$}, \forall n <= 10\\
			\text{$f_2$ does not dominate $f_4$}
		\end{eqnarray*}
	\Part{f}
	Statement: $f_3$ and $f_4$ are codominant
	\Solution
		$f_3 \in \BigOh{f_4}$ = $n^3 \in \BigOh{n^3}$,	
		$c >= n_0$,
		$ n_0 > 0$
		\begin{eqnarray*}
			\text{$n^3 <= cn^3$}\\
			\text{$c=2$}\\
			\text{$n^3 <= 2n^3$}, \forall n>0\\
			\text{$f_4$ does dominate $f_3$}
		\end{eqnarray*}
		$f_4 \in \BigOh{f_3}$ = $n^3 \in \BigOh{n^3}$,	
		$c >= n_0$,
		$ n_0 > 0$
		\begin{eqnarray*}
			\text{$n^3 <= cn^3$}\\
			\text{$c=2$}\\
			\text{$n^3 <= 2n^3$}, \forall n>0\\
			\text{$f_3$ does dominate $f_4$}
		\end{eqnarray*}
\Problem{2}
	\Part{a}
		Worst case
		$\BigOh{matmpy} = n^3$
		\Solution
			The first loop iterates over $n$ elements with object $i$. 

			The next nested loop iterates from $i$ to $n$ "$n$ times".

			 This can be expressed with:
			\begin{eqnarray}
				\sum_{i=1}^n i  & = & \frac{n(n+1)}{2} \\
			\end{eqnarray} 
			Which yields a $\BigOh{n^2}$. 

			The final nested loop iterates over $n$ elements.

			 Using the multiplicative rules of "Big Oh", $\BigOh{matmpy} = n^2 * n = n^3$

			 For function $t$ to dominate, and be reasonably tight $t=n^3$

			$matmpy \in \BigOh{t}$ = $n^3 \in \BigOh{n^3}$,	
			
			Proof:
				\begin{eqnarray*}
					\text{$n^3 <= cn^3$}\\
					\text{$c=2$}\\
					\text{$n^3 <= 2n^3$}, \forall n>0\\
					\text{$t$ does dominate $matmpy$}
				\end{eqnarray*}
	\Part{b}
		$\BigOh{mystery} = n$	
		\Solution
			Under the assumption that indents indicate nesting 

			The first loop doesn't do anything besides iterate over $n-1$ elements

			The next loop attempts to iterate from $i+1$ to $n$

			Since $i+1 = n$ after the first loop, the second loop never allows its nested loop to run
		
			$\BigOh{mystery} = n$

			For function $t$ to dominate, and be reasonably tight $t=n$
			
			$mystery \in \BigOh{t}$ = $n \in \BigOh{n}$

			Proof:
				\begin{eqnarray*}
					\text{$n <= cn$}\\
					\text{$c=2$}\\
					\text{$n<= 2n$}, \forall n>0\\
					\text{$t$ does dominate $mystery$}
				\end{eqnarray*}
	\Part{c}
		$\BigOh{veryodd} = n^2$	
		\Solution
			The first loop iterates over $n$ elements

			Using the additive rule of "Big Oh" the two nested loops merge to $n^2$

			Proof:
				\begin{eqnarray}
					\sum_{j=1}^n i  & = & \frac{n(n+1)}{2}  \\\
				\end{eqnarray} 
			The upper bound is defined when $i$ is odd, and yields $n^2 + n$ which simplifies to $\BigOh{n^2}$
			
			For function $t$ to dominate, and be reasonably tight $t=n^2$

			Proof:
				\begin{eqnarray*}
					\text{$n^2 <= cn^2$}\\
					\text{$c=2$}\\
					\text{$n^2 <= 2n^2$}, \forall n>0\\
					\text{$t$ does dominate $matmpy$}
				\end{eqnarray*}
	\Part{d}
	$\BigOh{recursive} = 2^n$
	\Solution
	For each recursive call, two more calls are generated. Total steps is the sum of $2^n$:
			\begin{eqnarray}
				\sum_{i=1}^n 2^i  & = & (2^{n+1}-1) \\
			\end{eqnarray} 
	Which yeilds $\BigOh{2^n}$

	For function $t$ to dominate, and be reasonably tight $t=2^n$

			Proof:
				\begin{eqnarray*}
					\text{$2^n <= c2^n$}\\
					\text{$c=2$}\\
					\text{$2^n <= 2(2^n)$}, \forall n>=0\\
					\text{$t$ does dominate $matmpy$}
				\end{eqnarray*}
	
\end{document}