%
% compile with pdflatex:
%   $ pdflatex homework-hw
% , yields homework-example.pdf
%    You might need to compile twice, if, e.g., you start using references
%

\documentclass[10pt,letterpaper,oneside]{article}
\usepackage[ascii]{inputenc}
\usepackage{graphicx}
\graphicspath{ {} }
\usepackage{amsmath,amsfonts,amssymb}
\usepackage[margin=1in]{geometry}
	\setlength{\parindent}{0em}
	\setlength{\parskip}{1em}



\newtheorem{theorem}{Theorem}

%%%% user definitions %%%%%%%%%%%%%%%%%%%%%%%

\newcommand{\Problem}[1]{\subsection*{Problem #1}}
\newcommand{\Part}[1]{\subsubsection*{Part #1}}
\newcommand{\Solution}{\subsubsection*{Solution}}

	% Forms for Big-Oh notation
\DeclareMathOperator{\Omicron}{O}
\DeclareMathOperator{\omicron}{o}

\newcommand{\BigOh}[1]{\Omicron(#1)}
\newcommand{\LittleOh}[1]{\omicron(#1)}
\newcommand{\BigOmega}[1]{\Omega(#1)}
\newcommand{\LittleOmega}[1]{\omega(#1)}
\newcommand{\BigTheta}[1]{\Theta(#1)}

	% Operators for dominance notation
\newcommand{\domeq}{\sim}
\newcommand{\domle}{\preceq}
\newcommand{\domlt}{\prec}
\newcommand{\domge}{\succeq}
\newcommand{\domgt}{\succ}

\newcommand\tab[1][1cm]{\hspace*{#1}}


%%%%%%%%  You edit stuff below this line  %%%%%%%%%%%%%%%%%%%%%%%%%%

\title{HW7 Theory}
\author{Alexander Kazantsev}
%\date{}  % uses today, by default

\begin{document}

\Problem{5.20}
for i in 1:n

\tab for z in i+1:n

\tab\tab if state[i].acceptance == state[z].acceptance

\tab\tab\tab Merge(state, i, z)

\tab\tab elif  state[i].input(0) == state[z].input(0)

\tab\tab\tab  Merge(state, i, z)

\tab\tab elif  state[i].input(1) == state[z].input(1)

\tab\tab\tab  Merge(state, i, z)
\Problem{2}

By definition of a undirected grpah, each edge $E$ is incident to exactly two vertices $\implies d_i = 2$

\begin{eqnarray*}
			\text{ $\sum_{i=1}^n d_i = 2m$}
\end{eqnarray*}
\Problem{3}
$\forall v_i$, $ i \in [1..n]$

let $x_i$ be out-degree of $v_i$ and $y_i$ be the in-degree of $v_i$
\begin{eqnarray*}
			\text{ $\sum_{i=1}^n x_i^2 - y_i^2  = \sum_{i=1}^n (x_i + y_i)(x_i - y_i)$}\\
			\text{ $x_i + y_i = n-1$}\\
			\text{ $\sum_{i=1}^n x_i^2 - y_i^2  = (n-1)\sum_{i=1}^n (x_i - y_i)$}\\
			\text{ $ \sum_{i=1}^n (x_i - y_i) = 0$}\\
\end{eqnarray*}

\maketitle
\end{document}